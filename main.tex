\documentclass{article}
% Preamble
\usepackage{hyperref}
\usepackage{booktabs}
\usepackage{xcolor}
\usepackage{placeins}
\usepackage{graphicx}
\usepackage{caption}
\usepackage{array}
\usepackage{subcaption}
\usepackage[most]{tcolorbox}
\usepackage{amsmath, amssymb, mathtools}
\usepackage{float}
\usepackage{placeins}  
\usepackage{mdframed} 
\usepackage[a4paper, top=1cm, 
left=1cm, right=2cm, bottom=2cm]{geometry}
\usepackage{tcolorbox}

\usepackage{minted}
\usepackage{listings}
\usepackage{xcolor}
\usepackage[export]{adjustbox}
\usepackage{tcolorbox}
\tcbuselibrary{skins, breakable}

\lstset{
  basicstyle=\ttfamily\small,
  backgroundcolor=\color{gray!10},
  frame=single,
  keywordstyle=\color{blue},
  commentstyle=\color{gray},
  stringstyle=\color{orange},
  showstringspaces=false,
  breaklines=true
}
\usepackage{listings}
\usepackage{xcolor}

\definecolor{codegray}{gray}{0.95}
\definecolor{matlabblue}{rgb}{0,0,0.6}

\lstdefinestyle{matlabstyle}{
    language=Matlab,
    backgroundcolor=\color{codegray},
    basicstyle=\ttfamily\footnotesize,
    keywordstyle=\color{matlabblue},
    commentstyle=\color{gray},
    stringstyle=\color{red},
    breaklines=true,
    showstringspaces=false,
    frame=single,
    tabsize=2,
    captionpos=b
}
\usepackage[table]{xcolor}
\usepackage{amsmath}
\usepackage{graphicx}

% \usepackage{draftwatermark}
% \SetWatermarkText{SYSE 511}            % <- Your watermark text here
% \SetWatermarkScale{1.5}                % <- Size of watermark
% \SetWatermarkColor[gray]{0.85}         % <- Light gray watermark

\begin{document}



\section*{Problem 1}
\subsection*{1a }
\[
\alpha = 7.14\ \text{[rad/s]},\quad \beta = 286.3\ \text{[rad/(V·s}^2\text{)]},\quad A = \begin{bmatrix} 0 & 1 \\ 0 & -7.14 \end{bmatrix},\quad B = \begin{bmatrix} 0 \\ 286.3 \end{bmatrix},\quad C = \begin{bmatrix} 1 & 0 \end{bmatrix},\quad Q = \begin{bmatrix} 1 & 0 \\ 0 & 0 \end{bmatrix}
\]


\begin{table}[h!]
\centering
\caption{LQR Gain and Closed-Loop Poles for $\rho = 1$ to $10$}
\label{tab:lqr_gains}
\begin{tabular}{c|cc|cc}
\toprule
$\rho$ & $K_1$ & $K_2$ & Pole 1 & Pole 2 \\
\midrule
1  & 1.000 & 0.062 & -12.486 & -12.486 \\
2  & 0.707 & 0.050 & -10.676 & -10.676 \\
3  & 0.577 & 0.043 & -9.767  & -9.767  \\
4  & 0.500 & 0.039 & -9.183  & -9.183  \\
5  & 0.447 & 0.036 & -8.761  & -8.761  \\
6  & 0.408 & 0.034 & -8.437  & -8.437  \\
7  & 0.378 & 0.032 & -8.176  & -8.176  \\
8  & 0.354 & 0.031 & -7.960  & -7.960  \\
9  & 0.333 & 0.029 & -7.776  & -7.776  \\
10 & 0.316 & 0.028 & -7.617  & -7.617  \\
\bottomrule
\end{tabular}
\end{table}

\begin{figure}[H]
    \centering    \includegraphics[width=0.5\textwidth]{p1a-gains-vs-rho.pdf}
    \caption{Variation of LQR Gains \( K_1 \) and \( K_2 \) as a Function of Control Weight \( \rho \) in State-Feedback Design }
    \label{fig:rho1-10}
\end{figure}
The model is designed to implement how the LQR state-feedback gains vary as a function of the control weighting parameter $\rho$, ranging from $1$ to $10$. It can be observed in Table~\ref{tab:lqr_gains} and Figure~\ref{fig:rho1-10}, both gains $K_1$ and $K_2$ decrease monotonically with increasing $\rho$, which indicates a more conservative control policy as higher emphasis is placed on minimizing control effort. The resulting closed-loop poles, which are listed in Table~\ref{tab:lqr_gains}, also shift closer to the imaginary axis as $\rho$ increases. Interestingly, the poles remain real and identical for all tested values of $\rho$, confirming critically damped behavior for this specific configuration of $Q$ and system dynamics. 

\newpage
\subsection*{1b}
Figure~\ref{fig:p1b_full} illustrates the evolution of the closed-loop poles. Here, the LQR control weight $\rho$ is modified over a wide range from $10^{-3}$ to $10^3$. As $\rho$ decreases, the poles move deeper into the left-half plane and gain larger imaginary components, which provides faster but potentially less stable responses. Conversely, as $\rho$ increases, the poles approach the imaginary axis, resulting in slower system dynamics with more conservative control behavior.
NOTE: Figure~\ref{fig:p1b_zoom}  highlights the region near the dominant poles for moderate values of $\rho$, and is for better visualization.
\begin{figure}[H]
    \centering
    \begin{subfigure}[t]{0.45\textwidth}
        \centering
        \includegraphics[width=\textwidth]{p1b.pdf}
        \caption{Full-range pole locus as $\rho \in [10^{-3}, 10^3]$}
        \label{fig:p1b_full}
    \end{subfigure}
    \hfill
    \begin{subfigure}[t]{0.45\textwidth}
        \centering
        \includegraphics[width=\textwidth]{p1bzoomed.pdf}
        \caption{Zoomed-in view of poles near the imaginary axis}
        \label{fig:p1b_zoom}
    \end{subfigure}
    \caption{Closed-loop pole trajectories as $\rho$ varies. 
    (a) shows the full pole locus over a wide sweep of $\rho$, while 
    (b) highlights the dominant poles near the imaginary axis.}
    \label{fig:rho_pole_locus}
\end{figure}

\subsection*{1c}
The effect assessment of the control weight $\rho$ on system performance is applied, we analyzed the angular position $\theta(t)$ and control effort $u(t)$ for two values of $\rho$: a defaults value of $0.5$ (which can be any other value) and an optimized value of $0.0231013$ obtained from the Matlab code solution. That optimal $\rho$ also satisfies the a 50~ms rise time criterion.
Figure~\ref{fig:theta_comparison} displays that a reduced $\rho$ enhances the system's reaction rate. The trajectory in subfigure~\ref{fig:theta_a} with $\rho = 0.5$ is noticeably slower and more gradual, while subfigure~\ref{fig:theta_b} indicates that with $\rho = 0.0231013$, the system achieves faster convergence and better damping. Nonetheless, this enhanced performance implies a greater control effort, as seen in Figure~\ref{fig:control_comparison}. Subfigure~\ref{fig:control_a} illustrates the moderate voltage levels for $\rho = 0.5$, whereas subfigure~\ref{fig:control_b} demonstrates the considerably greater control magnitude necessary for the optimum $\rho$.

\begin{figure}[H]
    \centering
    \begin{subfigure}[b]{0.45\textwidth}
        \centering
        \includegraphics[width=\textwidth]{p1c_theta05.pdf}
        \caption{$\theta(t)$ with $\rho = 0.5$}
        \label{fig:theta_a}
    \end{subfigure}
    \hfill
    \begin{subfigure}[b]{0.45\textwidth}
        \centering
        \includegraphics[width=\textwidth]{p1c_theta.pdf}
        \caption{$\theta(t)$ with optimal $\rho = 0.0231013$}
        \label{fig:theta_b}
    \end{subfigure}
    \caption{Angular position $\theta(t)$ comparison for two values of $\rho$. (a) Higher $\rho = 0.5$ yields a slower response. (b) Optimized $\rho = 0.0231013$ achieves faster and well-damped behavior.}
    \label{fig:theta_comparison}
\end{figure}
\begin{figure}[H]
    \centering
    \begin{subfigure}[b]{0.45\textwidth}
        \centering
        \includegraphics[width=\textwidth]{p1c_control05.pdf}
        \caption{$u(t)$ with $\rho = 0.5$}
        \label{fig:control_a}
    \end{subfigure}
    \hfill
    \begin{subfigure}[b]{0.45\textwidth}
        \centering
        \includegraphics[width=\textwidth]{p1c_control.pdf}
        \caption{$u(t)$ with optimal $\rho = 0.0231013$}
        \label{fig:control_b}
    \end{subfigure}
    \caption{Control input $u(t)$ comparison. (a) Lower effort for $\rho = 0.5$. (b) Optimized $\rho = 0.0231013$ requires higher initial control magnitude to meet the rise time constraint.}
    \label{fig:control_comparison}
\end{figure}


 
\subsection*{1d [EC]}
\begin{figure}[H]
    \centering

    \begin{subfigure}[t]{0.45\textwidth}
        \centering
        \includegraphics[width=\textwidth]{p1d-finite-horizon-state.pdf}
        \caption{State response $\theta(t)$ for finite and infinite horizons}
        \label{fig:p1d_theta}
    \end{subfigure}
    \hfill
    \begin{subfigure}[t]{0.45\textwidth}
        \centering
        \includegraphics[width=\textwidth]{p1d-finite-horizon-control.pdf}
        \caption{Control input $u(t)$ for finite and infinite horizons}
        \label{fig:p1d_control}
    \end{subfigure}

    \caption{Comparison between finite-horizon and infinite-horizon LQR solutions. 
    (a) shows the angular position $\theta(t)$ under three controllers: one infinite-horizon and two finite-horizon designs (50 ms and 200 ms). 
    (b) shows the corresponding control inputs. Shorter horizons yield more time-varying and aggressive control.}
    \label{fig:p1d_finite_lqr}
\end{figure}
To compare finite-horizon and infinite-horizon LQR performance, we solved the time-varying Riccati differential equation backward in time from $t_f$ for two horizon lengths: 50\,ms and 200\,ms. Figure~\ref{fig:p1d_finite_lqr}a shows that shorter finite horizons (e.g., 50\,ms) produce responses that initially diverge from the infinite-horizon behavior via aggressive control efforts. Longer horizons such as 200\,ms mathc more closely with the infinite-horizon solution over time. Figure~\ref{fig:p1d_finite_lqr}b highlights the control effort $u(t)$: with a 50\,ms horizon, the controller acts more forcefully early on, while the 200\,ms solution closely tracks the infinite-horizon strategy.


\newpage
\section*{Problem 2}
\subsection*{2a and 2b}
Here 2a and 2b solutions are provided together, and the final result is in the conclusion section (Table~\ref{tab:comparison} ). A simulation-based optimization approach was implemented using \texttt{fmincon}, with spline-interpolated control discretized over varying numbers of time points \(n_t\). For each case, the numerical values of the cost \(J\), final control \(u(t_f)\), and final state \(x(t_f)\) were computed via the code and presented in Table~\ref{tab:problem2}. Before analyzing that let's write the Session 9 results with the given fixed values:
\[
J^* = -\frac{2}{3} \approx -0.66667, \quad u^*(1) = \frac{4}{9} \approx 0.4444, \quad x^*(1) = \left(\frac{2}{3}\right)^2 = 0.4444.
\]
Table~\ref{tab:problem2} introduces the simulation results for \(n_t = 3\) to \(35\). A combined absolute error metric I used is:
\[
\text{Error} = |J - J^*| + |u(t_f) - u^*(1)| + |x(t_f) - x^*(1)|.
\]
Emphasizing just on Objective function value or providing different weights to the \(J\),\(u(t_f)\),\(x(t_f)\) would be another approach. I just did not want to make it easier or more complicated.
\begin{table}[H]
\centering
\caption{Simulation results for varying time discretization points \(n_t\).}
\label{tab:problem2}
\begin{tabular}{|c|c|c|c|c|}
\hline
\textbf{$n_t$} & \textbf{J} & \textbf{$u(t_f)$} & \textbf{$x(t_f)$} & \textbf{Total Error} \\
\hline
3  & $-0.66897$ & $0.39644$ & $0.44099$ & $0.05376$ \\
4  & $-0.66737$ & $0.41151$ & $0.44338$ & $0.03470$ \\
5  & $-0.66696$ & $0.41914$ & $0.44400$ & $0.02603$ \\
\hline
\textbf{\textcolor{blue}{11}} & \textbf{\textcolor{blue}{$-0.66669$}} & \textbf{\textcolor{blue}{$0.43383$}} & \textbf{\textcolor{blue}{$0.44441$}} & \textbf{\textcolor{blue}{$0.01066$}} \\
12 & $-0.66670$ & $0.43461$ & $0.44435$ & $0.00996$ \\
13 & $-0.66670$ & $0.43625$ & $0.44443$ & $0.00824$ \\
\hline
34 & $-0.66986$ & $0.44746$ & $0.44512$ & $0.00688$ \\
35 & $-0.67023$ & $0.44867$ & $0.44527$ & $0.00861$ \\
\hline
\end{tabular}
\end{table}
From this table, it can be seen that the error becomes minimal after the \(n_t = 11\), and I chose the discretization level of \(n_t = 11\). The combined error is relatively low, and the values of \(J\), \(u(t_f)\), and \(x(t_f)\) are all in close agreement with its Session 9 results (Table~\ref{tab:comparison} ).
The Figure~\ref{fig:convergence} visualizes how the objective value, final control input, and final state evolve as a function of \(n_t\). We see more monotone-level plots in the top-left (the objective \(J\)) and similarly, the final state \(x(t_f)\).  Bottom-top right plots have more exponential changes in them. I believe these visual trends reinforce the numerical results in Table~\ref{tab:problem2}.
\begin{figure}[h]
    \centering
    \includegraphics[width=0.85\textwidth]{mine_problem_convergence.pdf}
    \caption{
    The figure shows how the objective \(J\), final control \(u(t_f)\), and state \(x(t_f)\) change with discretization \(n_t\), along with the total error. Best accuracy is chosen as \(n_t = 11\) .
}

    \label{fig:convergence}
\end{figure}







\begin{figure}[H]
    \centering
    \begin{subfigure}[t]{0.48\textwidth}
        \centering
        \includegraphics[width=\textwidth]{p2control.pdf}
        \caption{Optimal control signal $u(t)$}
        \label{fig:p2a_control}
    \end{subfigure}
    \hfill
    \begin{subfigure}[t]{0.48\textwidth}
        \centering
        \includegraphics[width=\textwidth]{p2state.pdf}
        \caption{Resulting state trajectory $x(t)$}
        \label{fig:p2a_state}
    \end{subfigure}
    \caption{Optimal control results computed using \texttt{fmincon} with $N = 11$ time points. 
    (a) shows the control input $u(t)$; (b) shows the corresponding state trajectory $x(t)$.}
    \label{fig:p2a_results}
\end{figure}

Figure~\ref{fig:p2a_results} illustrates the optimized control input and resulting system response when $N = 11$. In Fig.~\ref{fig:p2a_control}, the control signal $u(t)$ decreases smoothly over time in a stepwise fashion due to discretization, whereas Fig.~\ref{fig:p2a_state} shows the corresponding state trajectory $x(t)$, which declines monotonically as expected. 

[CONCLUSION] To conclude, in my opinion  $N = 11$ time points are needed for an accurate solution, which can be seen from the comparison Table~\ref{tab:comparison} 



\begin{table}[h!]
\centering
\begin{tabular}{|c|c|c|c|}
\hline
\textbf{Source} & $J$ & $x(1)$ & $u(1)$ \\
\hline
Session 9 Result & $-0.66666$ & $0.44444$ & $ 0.44444$ \\
\hline
My Result & $-0.66669$ & $0.43383$ & $0.44441$ \\
\hline
\end{tabular}
\caption{Comparison of Session 9 Analytical Results and My Numerical Results at $t = 1$}
\label{tab:comparison}

\end{table}

\subsection*{2c}
I also did a similar approach here, like I applied in previous cases, and from the data analytics results, it seems in this constrained case it takes at least \( N = 54 \) time points for the model to find the a reasonably accurate and smooth solution.

\begin{table}[H]
\centering
\caption{Constrained Optimal Control Problem Results}
\label{tab:constrained_ocp}
\begin{tabular}{|c|c|c|c|c|}
\hline
\textbf{$n_t$} & \textbf{J} & \textbf{$u(t_f)$} & \textbf{$x(t_f)$} & \textbf{Total Error} \\
\hline
15 & $-0.66297$ & $0.45690$ & $0.46570$ & $0.03741$ \\
16 & $-0.66299$ & $0.45761$ & $0.46558$ & $0.03798$ \\
17 & $-0.66300$ & $0.45936$ & $0.46579$ & $0.03993$ \\
\hline
\multicolumn{5}{|c|}{\vdots} \\
\hline
50 & $-0.66513$ & $0.48532$ & $0.47463$ & $0.07261$ \\
51 & $-0.66789$ & $0.50225$ & $0.48499$ & $0.09958$ \\
52 & $-0.66657$ & $0.49537$ & $0.47061$ & $0.07719$ \\
53 & $-0.66663$ & $0.49676$ & $0.46529$ & $0.07320$ \\
\textbf{\textcolor{blue}{54}} &
\textbf{\textcolor{blue}{$-0.66660$}} &
\textbf{\textcolor{blue}{$0.49799$}} &
\textbf{\textcolor{blue}{$0.46214$}} &
\textbf{\textcolor{blue}{$0.07131$}} \\
55 & $-0.66707$ & $0.50144$ & $0.45930$ & $0.07226$ \\
\hline
\end{tabular}
\end{table}
Figure~\ref{fig:p2c_results} illustrates this behavior. In Fig.~\ref{fig:p2c_control}, the control signal maintains a constant maximum value before experiencing a swift decrease towards the conclusion. This transition introduces steep gradients in the control profile, which are harder to capture accurately with a coarse time discretization. Figure \ref{fig:p2c_state} illustrates the resultant state trajectory, which maintains a smooth and monotonic nature despite the more abrupt control changes Due to this change in problem structure, a finer discretization is needed to resolve the sharp change in control.

\begin{figure}[H]
    \centering
    \begin{subfigure}[t]{0.45\textwidth}
        \centering
        \includegraphics[width=\textwidth]{p2c_control.pdf}
        \caption{Optimal control $u(t)$ with constraint $u(t) \leq 0.55$}
        \label{fig:p2c_control}
    \end{subfigure}
    \hfill
    \begin{subfigure}[t]{0.45\textwidth}
        \centering
        \includegraphics[width=\textwidth]{p2c_state.pdf}
        \caption{Resulting state trajectory $x(t)$}
        \label{fig:p2c_state}
    \end{subfigure}
    \caption{Constrained optimal control results with path constraint $u(t) \leq 0.55$. 
    (a) shows the control profile with a flat segment due to the active constraint, 
    followed by a sharp drop. (b) shows the corresponding state response.}
    \label{fig:p2c_results}
\end{figure}








\end{document}

